\section{Introduction}
\label{sec:intro}
% Why ultra-high speed bandwidth estimation
Bandwidth estimation for ultra-high speed network has been actively studied
during recent years, due to the coming availability next generation network
such as Google fiber~\cite{GoogleFiber}.

% challenge
While existing bandwidth estimation algorithms work well on 100~Mbps networks,
they do not scale well to gigabit and higher network speeds~\cite{shriram2005}.
Bandwidth estimation on ultra-high speed network is challenging because smaller
inter-packet gaps are needed for probing higher bandwidth, where such
fine-scale gaps are more vulnerable against buffering related noise at the
shared resources and at end hosts.

% state-of-art method
Recently,~\cite{Yin2014} has proposed a novel smoothing scheme, Buffering-aware
Spike Smoothing (BASS), which helps significantly in scaling bandwidth
estimation to ultra-high speed networks. By carefully observing the receiving
gaps from the probing sequences, they first extract each buffering event by
detecting spikes in the receiving signals. Then by smoothing each buffering
event, they are able to recover the underlying pattern for bandwidth
estimation.

% machine learning perspectives
Though~\cite{Yin2014} had successfully pushed the bandwidth prediction problem
onto much higher-speed networks, hand-crafted smoothing strategies may not take
full advantages of high volume of available data, and may neglect some hidden
subtle features of the probing streams. Thus we want to re-visit the same
problems from a different perspective. Specifically, we
formulate this problem with a regression model, which maps the probing stream
statistics to an available bandwidth prediction.

Machine learning is the process of
discovering the underlying pattern amongst various input signals and a single
or structured output. Normally, we call the task a classification problem when
the output is discrete labels, and a regression problem if the output is
continuous value\cite{bishop2006pattern}, where the bandwidth estimation is
clearly a regression problem. Regression methods are powerful when plenty of
observation are available for training and validating the model, which makes it
ideal for bandwidth estimation task, since we could easily obtain thousands of
measurements from the testbed.

Our main contributions are summarized below:
\begin{itemize}
\item We propose a full modular pipeline for bandwidth estimating on ultra-high
  speed networks, consisting of data cleaning, feature engineering, model
  training and model evaluation;
\item We experimentally show that our methods can effectively reduce error rate
  to 1.82\% compared with 6.97\% from~\cite{Yin2014}. To the best of our
  knowledge, we are the first to demonstrate the effectiveness of applying
  regression to solve the bandwidth estimation problem even for ultra-high
  speed networks.
\end{itemize}

In the rest of this paper, we give a quick review of several regression models
in use in Section~\ref{sec:method} and feature engineering in
Section~\ref{sec:feature_engineering} which is vital for handling noises in
input signals. Our experimental findings are summarized and analyzed in
Section~\ref{sec:experiment}. We conclude our work in
Section~\ref{sec:conclusion} and discussed a few future directions in the hope
to inspire more works to introduce machine learning techniques to this field in
Section~\ref{sec:future_work}.
