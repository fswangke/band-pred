\section{Experiment}
\label{sec:experiment}

\subsection{Feature Engineering}
\label{sub:feature_engineering}

\subsubsection{Sequence Smoothing}
\label{ssub:sequence_smoothing}

\begin{figure}[htpb]
   \centering
   \subfloat[Send gaps]{
      \includegraphics[width=0.45\linewidth]{figures/smooth_send.pdf}
   }
   \quad
   \subfloat[Received gaps]{
      \includegraphics[width=0.45\linewidth]{figures/smooth_recv.pdf}
   }
   \caption{Smoothing raw packet gaps.}
   \label{fig:smoothing}
\end{figure}

\subsubsection{Fourier Transformation}
\label{ssub:fourier_transformation}

\begin{figure}[htpb]
   \centering
   \subfloat[Send gaps]{
      \includegraphics[width=0.45\linewidth]{figures/fft_send.pdf}
   }
   \quad
   \subfloat[Received gaps]{
      \includegraphics[width=0.45\linewidth]{figures/fft_recv.pdf}
   }
   \caption{Fourier transformation on raw packet gaps.}
   \label{fig:fft}
\end{figure}

\subsection{Parameter Setting}
\label{sub:parameter_setting}



\subsection{Benchmark}
\label{sub:benchmark}

\subsubsection{Computation Performance}
\label{ssub:computation_performance}



\begin{table}[htpb]
   \centering
   \caption{Average processing time over 12279 streams.}
   \label{tab:timing}
   \begin{tabular}{|c|r|r|r|r|r|r|}
      \hline
      \multirow{2}{*}{Algorithms} & \multicolumn{2}{c|}{Feature extraction} &
      \multicolumn{4}{c|}{Training} \\ \cline{2-7}
                       & Raw   & Smooth & Raw            & Raw FFT        & Smooth         & Smooth FFT \\ \hline
      AdaBoost         & 15.71 & 14.94  & 2026.13        & 35766.15       & 13109.74       & 32991.03\\
      Elastic net      & 16.60 & 16.06  & 151.96         & 577.30         & 121.73         & 537.47\\
      Gradient boost   & 12.55 & 12.52  & 8090.82        & 45908.08       & 9695.63        & 46406.90\\
      Lasso            & 16.23 & 16.14  & 145.09         & 507.52         & 139.83         & 567.07\\
      NNR              & 15.78 & 15.59  & \textbf{18.75} & \textbf{33.29} & \textbf{18.34} & \textbf{35.41}\\
      Random forest    & 15.80 & 15.61  & 3139.64        & 9236.24        & 3207.93        & 9380.83\\
      \hline
   \end{tabular}
\end{table}

\subsubsection{Accuracy}
\label{ssub:accuracy}
\begin{table}[htpb]
   \centering
   %6.97
   \caption{Average accuracy of different estimators over differnt features. We
      evaluated the average accuracy over all datasets, with 12279 probing
      sequences in total. We include accuracy increases compared with baseline
      from \cite{Yin2014} in parentheses. \cite{Yin2014} has an average
      accuracy of 6.97\%.}
   \label{tab:label}
   \begin{tabular}{|c|c|c|c|c|}
      \hline
      Algorithms     & Raw                      & Raw FFT                  & Smooth                   & Smooth FFT \\ \hline
      AdaBoost       & 2.82\%(59.43\%)          & 2.38\%(65.72\%)          & 2.29\%(67.14\%)          & 2.18\%(68.65\%)\\
      Elastic net    & 2.30\%(66.95\%)          & 2.19\%(68.58\%)          & 2.16\%(68.97\%)          & 1.96\%(71.87\%)\\
      Gradient boost & 2.41\%(65.34\%)          & \textbf{1.96\%}(71.75\%) & \textbf{1.89\%}(72.88\%) & \textbf{1.82\%}(73.77\%)\\
      Lasso          & 2.70\%(61.18\%)          & 2.37\%(65.89\%)          & 2.25\%(67.65\%)          & 2.03\%(70.78\%)\\
      NNR            & \textbf{2.29\%}(67.11\%) & 2.29\%(67.11\%)          & 2.17\%(68.83\%)          & 2.17\%(68.83\%)\\
      Random forest  & 2.35\%(66.26\%)          & 2.01\%(71.06\%)          & 2.05\%(70.47\%)          & 1.89\%(72.86\%)\\
      \hline
   \end{tabular}
\end{table}

\begin{figure}[htpb]
   \centering

   \subfloat[AdaBoost]{
      \includegraphics[width=0.3\linewidth]{figures/histogram_adaboost.pdf}
   }
   \quad
   \subfloat[Elastic net]{
      \includegraphics[width=0.3\linewidth]{figures/histogram_elastic_net.pdf}
   }
   \quad
   \subfloat[Gradient boost]{
      \includegraphics[width=0.3\linewidth]{figures/histogram_gradient_boost.pdf}
   }
   \\
   \subfloat[Lasso]{
      \includegraphics[width=0.3\linewidth]{figures/histogram_lasso.pdf}
   }
   \quad
   \subfloat[NNR]{
      \includegraphics[width=0.3\linewidth]{figures/histogram_nnr.pdf}
   }
   \quad
   \subfloat[Random forest]{
      \includegraphics[width=0.3\linewidth]{figures/histogram_random_forest.pdf}
   }
   \caption{Relative error histograms of different estimators. We ran learning
      based estimators on FFT transformed smooth sequences. Example dataset has
      range 50, rates 8, packets 16.}
   \label{fig:name}
\end{figure}

\begin{figure}[htpb]
   \centering
   \subfloat[Relative error]{
      \includegraphics[width=0.45\linewidth]{figures/error_exp4_range50_rates8_pkts16.pdf}
   }
   \quad
   \subfloat[Standard derivation]{
      \includegraphics[width=0.45\linewidth]{figures/std_exp4_range50_rates8_pkts16.pdf}
   }
   \caption{Dataset: range 50, rates 8, packets 16}
   \label{fig:exp4}
\end{figure}

\begin{figure}[htpb]
   \centering
   \subfloat[Relative error]{
      \includegraphics[width=0.45\linewidth]{figures/error_exp5_range50_rates2_pkts64.pdf}
   }
   \quad
   \subfloat[Standard derivation]{
      \includegraphics[width=0.45\linewidth]{figures/std_exp5_range50_rates2_pkts64.pdf}
   }
   \caption{Dataset: range 50, rates 2, packets 64}
   \label{fig:exp5}
\end{figure}

\begin{figure}[htpb]
   \centering
   \subfloat[Relative error]{
      \includegraphics[width=0.45\linewidth]{figures/error_exp6_range50_rates3_pkts43.pdf}
   }
   \quad
   \subfloat[Standard derivation]{
      \includegraphics[width=0.45\linewidth]{figures/std_exp6_range50_rates3_pkts43.pdf}
   }
   \caption{Dataset: range 50, rates 3, packets 43}
   \label{fig:exp6}
\end{figure}

\begin{figure}[htpb]
   \centering
   \subfloat[Relative error]{
      \includegraphics[width=0.45\linewidth]{figures/error_exp7_range50_rates6_pkts21.pdf}
   }
   \quad
   \subfloat[Standard derivation]{
      \includegraphics[width=0.45\linewidth]{figures/std_exp7_range50_rates6_pkts21.pdf}
   }
   \caption{Dataset: range 50, rates 6, packets 21}
   \label{fig:exp7}
\end{figure}

\begin{figure}[htpb]
   \centering
   \subfloat[Relative error]{
      \includegraphics[width=0.45\linewidth]{figures/error_Nov9_range50_rates4_pkts32.pdf}
   }
   \quad
   \subfloat[Standard derivation]{
      \includegraphics[width=0.45\linewidth]{figures/std_Nov9_range50_rates4_pkts32.pdf}
   }
   \caption{Dataset: range 50, rates 4, packets 32}
   \label{fig:nov9}
\end{figure}
